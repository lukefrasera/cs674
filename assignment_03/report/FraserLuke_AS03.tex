\documentclass[letter]{article}
\usepackage[utf8]{inputenc}
\usepackage[top=1in, bottom=1in, left=1in, right=1in]{geometry}

%opening
\title{Fourier Transform Experimentation \\ Report}
\author{Luke Fraser}

\begin{document}

\maketitle

\begin{abstract}
In this assignment the use of the FFT (Fast Fourier Transofrm) is tested and analyzed to show its capabilties on different types of data. As well the theoretical aspects that were proven and understood in class are shown to appear in the experiemtnal results as well. The use of the Fourier Transform is far reaching and allows for a faster solution of the convolution in special cases. As well all of the experiments in this paper are implimented with the 1D FFT to show that a 2D FT can be computed through the use of a 1D FT.
\end{abstract}

\section{Experiment 1}
\subsection{Part A}
\subsection{Part B}
\subsection{Part C}
\section{Experiment 2}
\subsection{Part A}
\subsection{Part B}
\subsection{Part C}
\section{Experiment 3}
\subsection{Part A}
\subsection{Part B}

\end{document}
